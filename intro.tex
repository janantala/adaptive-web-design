% %%%%%%%%%%%%%%%%%% FRONT PAGE %%%%%%%%%%%%%%%%%%%%%%%%%%%%

\pagenumbering{roman}

\setlength{\parindent}{0cm}

\thispagestyle{empty}

\begin{center}
%\vglue-2.5cm
%\scalebox{0.8}{\includegraphics{img/fiit.pdf}}
\begin{LARGE}
\textmd{
Slovenská technická univerzita v Bratislave\\
\vspace*{0.2cm}
Fakulta informatiky a informačných technológií  
}
\end{LARGE}

\vspace*{1.0cm}
\begin{Large}
% \textmd{FIIT-5212-64295}
\end{Large}

%\vglue-0.3cm
%\hrulefill
\end{center}

\vspace{5.5cm}

\begin{center}
{\Large \textmd{{Bc. Ján Antala}}}
\end{center}

\vspace{0.1cm}
\begin{huge}
\begin{center}
\textsc{Adaptívny web dizajn}
\end{center}
\end{huge}

\vspace{0.5cm}
\begin{center}
{\Large{\textmd{Diplomová práca}}}\\
\end{center}

\vspace{5.5cm}

\begin{flushleft}
\large{Študijný program: Informačné systémy} \\
\large{Študijný odbor: 9.2.6 Informačné systémy} \\
\large{Miesto vypracovania:  Ústav aplikovanej informatiky, FIIT STU, Bratislava} \\
\large{Vedúci práce: doc. Ing. Michal Čerňanský PhD.} \\
\vspace{1.0cm}
\large{máj 2014} \\
\end{flushleft}


%\newpage
%\thispagestyle{empty}
%\mbox{}

\newpage


%%%%%%%%%%%%%%%%%%% ZADANIE - DUMMY PAGE %%%%%%%%%%%%%%%%%%%%%%%
%\pagebreak
%
%\section*{Zadanie}
%\addcontentsline{toc}{section}{Zadanie}
%zadanie\ldots\ldots\ldots\ldots\ldots\ldots

%%%%%%%%%%%%%%%%%%% ANOTACIE  %%%%%%%%%%%%%%%%%%%%%%%%%%%%


\section*{ANOTÁCIA}
% \addcontentsline{toc}{section}{ANOTÁCIA}
\thispagestyle{empty}
Slovenská technická univerzita v Bratislave\\
FAKULTA INFORMATIKY A INFORMAČNÝCH TECHNOLÓGII\\
Študijný program: INFORMAČNÉ SYSTÉMY
\newline

Autor: Bc. Ján Antala\\
Diplomová práca: Adaptívny web dizajn.\\ 
Vedúci diplomovej práce: doc. Ing. Michal Čerňanský PhD. \\
Máj, 2014
\newline

Používanie mobilných zariadení rapídne rastie a tie so sebou prinášajú nie len rôzne veľkosti displejov a nové interakčné spôsoby, ale aj nové druhy pripojenia na internet. Ale na web nemajú prístup len smartfóny a klasické počítače. Sú tu smartfóny, tablety, e-čítačky, netbooky, hodinky, televízie, phablety, herné konzoly a množstvo ďalších. Je preto nevyhnutné prispôsobiť webové stránky rôznym zariadeniam a novým trendom v oblasti interakcie. Rapídne sa vyvíja niekoľko spôsobov dizajnovania webu. My prinášame spôsob adaptácie webových komponentov založený na možnostiach poskytovanými zariadeniami a externých podmienkach. Uvádzame nové spôsoby interakcie s webovými aplikáciami založené na rozpoznaní reči, pohybu alebo rotácie zariadenia.

\newpage

\section*{ANNOTATION}
% \addcontentsline{toc}{section}{ANNOTATION}
\thispagestyle{empty}
Slovak University of Technology Bratislava\\
FACULTY OF INFORMATICS AND INFORMATION TECHNOLOGIES\\
Degree Course: INFORMATION SYSTEMS
\newline

Author: Bc. Ján Antala\\
Diploma Thesis: Adaptive web design. \\
Supervisor: doc. Ing. Michal Čerňanský PhD. \\
2014, May
\newline

Mobile device penetration grows rapidly and it brings not only dif- ferent display sizes with new human interaction methods, but also high-latency networks. But there are not just smartphones and desktop computers. We now have Web-enabled smartphones, tablets, e-readers, netbooks, watches, TVs, phablets, notebooks, game consoles, cars and more. So it is necessary to adapt web pages for different devices and new interaction methods. Several new methods for designing web are rapidly evolving. We present a new way of web components adaptation based on a device features and external conditions in our work. We introduce new possibilities of interaction with the web applications based on speech recognition, motion or device rotation.


%%%%%%%%%%%%%%%%%%% PREHLASENIE %%%%%%%%%%%%%%%%%%%%%%%%%%%%
\pagebreak
\thispagestyle{empty}
\vglue0pt\bigskip\vfil
 \noindent\textbf{Čestné prehlásenie}
\newline\newline
Čestne prehlasujem, že záverečnú prácu som vypracoval samostatne, s použitím uvedenej
literatúry a na základe svojich vedomostí a znalostí.
\newline\newline
\noindent Bratislava, máj 2014 \hfil
\begin{tabular}[t]{c}
\hbox to 50mm {\dotfill} \\ \textit{\small Vlastnoručný podpis}
\end{tabular}\qquad\linebreak

%%%%%%%%%%%%%%%%%%% PODAKOVANIE %%%%%%%%%%%%%%%%%%%%%%%%%%%%
\pagebreak
\thispagestyle{empty}
\vglue0pt\bigskip\vfil
 \noindent\textbf{Poďakovanie}
\bigskip

Týmto sa chcem poďakovať najmä vedúcemu práce, doc. Ing. Michalovi Čerňanskému, PhD.,
za jeho cenné rady, pripomienky a venovaný čas pri konzultáciách. Vďaka patrí taktiež mojej rodine a všetkým kontribútorom podiaľajúcich sa na zlepšení vytvoreného softvéru.
\pagebreak

%%%%%%%%%%%%%%%%%%% OBSAH, (LOT, LOF)  %%%%%%%%%%%%%%%%%%%%%%%%%%%
\newpage
\setcounter{page}{1}
\thispagestyle{empty}
\setcounter{tocdepth}{3}
\tableofcontents %% tu sa generuje obsah
\newpage 
\thispagestyle{empty}
\listoftables
\newpage
\thispagestyle{empty}
\listoffigures 
\newpage
