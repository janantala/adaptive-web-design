\section{Úvod} % (fold)
\label{sec:_vod}

% Postupom času sa vyvinuli rôzne spôsoby ako prispôsobiť web aj na mobilné zariadenia. Medzi najpoužívanejšia techniky patria ,,\textit{Responsive design}'', ,,\textit{Mobile-first responsive design}'', ,,\textit{Progressive enhancement}'' a ,,\textit{Server-side adaptation}''\cite{mobiforge} .

% Najlepší spôsob prispôsobenia obsahu je však taký, ktorý pokrýva všetky zariadenia od najmenších mobilných po veľké televízne obrazovky a nevytvára pre rôzne zariadenia vlastné stránky. Tieto podmienky z časti spĺňa technika ,,\textit{Mobile-first responsive design}'' založená na používaní flexibilného vzhľadu stránky, flexibilných obrázkov a media queries \cite{responsive, mediaqueries}, pričom sa postupuje aj od malých zariadení k veľkým obrazovkám. Táto technika však nemusí byť vždy úplne postačajúca, a preto sa hodí využiť kombináciu rôznych spôsobov.

S príchodom mobilných zariadení, ktoré postupne nahradzujú klasické počítače, vznikol problém pri správnom zobrazovaní webového obsahu. Web na ne nebol pripravený, a tak sa im ponúkala verzia stránky pre klasické počítače.

Len za posledných pár rokov sa vo svete predalo viac ako 1 bilión mobilných zariadení, 1.038 bilióna celkovo \cite{bilion}. Mobilné telefóny a tablety sa stali ešte viac personálnymi a používajú sa neustále počas celého dňa pri rôznych činnostiach \cite{mobileuse, smarthopenseveryday, tabletuse}. Ich predajnosť sa zvýšuje každým dňom. V porovnaní vývoja trhového podielu osobných počítačov typu WINTEL a mobilných zariadení Apple a Android za posledné roky je tento trend ešte viac viditeľný.

Postupom času sa vyvinuli rôzne spôsoby ako prispôsobiť web aj na mobilné zariadenia. Najlepší spôsob prispôsobenia obsahu je taký, ktorý pokrýva všetky zariadenia od najmenších mobilných po veľké televízne obrazovky a nevytvára pre rôzne zariadenia vlastné samostané stránky.

Ignorovanie webu na mobilných zariadeniach väčšinou spoločností však vedie k vytváraniu samostatných natívnych aplikácii pre každú platformu. Okrem vedúceho postavenia Androidu a iOS existuje množstvo ďalších platforiem, pre ktorú treba vytvoriť vlastnú aplikáciu, a tým sa vývoj predražuje.

Nevýhodou natívnych aplikácii je, že neotvárajú webové odkazy a stále nemáme istotu, že si ich používateľ stiahne a nainštaluje. Taktiež vzniká problém pri aktualizáciách, používateľ ju músí manuálne spustiť. Nestačí len otvoriť aplikáciu, ktorá bude automaticky obsahovať najnovšiu verziu ako web. S tým je spojený aj problém s ich udržiavaním.

Práve tu je neoddeliteľná súčasť webu a mobilných zariadení. V súčasnosti populárne sociálne siete, ale aj emaily či qr kódy obsahujú množstvo odkazov na web. Na rovnaké odkazy je však možné pristúpiť aj z klasických počítačov. Je tak dôležité, aby sa obsah používateľom zobrazil správne bez ohľadu na to, na akom zariadení k nemu pristupujú.

Optimalizácia webového obsahu pre mobilné zariadenia má veľmi krátku históriu. Dnes však už existujú základné vzory, podľa ktorých je možné prispôsobiť zobrazenie webového obsahu na ich malé displeje \cite{mobilebookpatterns, navigation}. 

Problémom je, že neustále rastú rozmery mobilných zariadení, ale aj veľké televízne obrazovky sa stávajú prístupovým bodom k webovému obsahu. Len za posledné 3 mesiace bolo predaných 29\% android zariadení s obrazovkou väčšou ako 4.5 palca \cite{bigscreen} a k podobnému trendu pristupujú aj iní výrobcovia.

Optimalizácia webového obsahu zariadeniam však neznamená len jeho vizuálne prispôsobene rôznym veľkostiam displejom. Nemenej dôležitým prvkom je aj pripojiteľnesť zariadenia na sieť s cieľom čo najrýchlejšieho stihnutia, zobrazenia stránky a šetrenia používateľových prenášaných dát a financií.

Nielen práve preto je vhodné použiť princípy adaptívneho web dizajnu. Medzi jeho hlavné charakteristiky patria všadeprítomnosť, flexibilita, výkonnosť, rozšíriteľnosť a priateľskosť k budúcnosti \cite{adaptive}. V súčasnosti nevieme povedať aké zariadenia sa budú predávať o pár rokov, aké budú mať vlastnosti, ale s celkom veľkou pravdepodobnosťou budú obsahovať webový prehliadač. 

Výzvou sa tak stáva nie len vytvorenie celkového používateľského rozhrania, ale aj jednotlivých prvkov ako je navigácia či ovládacie prvky, ktoré by pomohli správne zobraziť obsah na malých zariadeniach a zároveň aby sa obsah prispôsobil aj tabletom, klasickým počítačom či pre veľké obrazovky televíznych príjmačov. Keďže rozlíšenia mobilných zariadení sa začínajú prelínať s klasickými počítačmi a aj klasické počítače pridávajú nové spôsoby ovládania dotykom, je dôležité rozoznať spôsob ovládania zariadenia a prispôsobiť navigáciu a obsah na dotyk alebo klávesnicu a myš. Rovnako je potrebné rozoznať aj internetové pripojenie zariadenia a automaticky mu odovzdať taký obsah, aby sa mu čo najrýchlejšie načítal a aby to používateľa nestálo zbytočný čas a financie za prenášané dáta.



% section _vod (end)


% section adapta_n_techniky (end)

\section{Adaptácia} % (fold)
\label{sec:adapt_cia}

\subsection{Možnosti prispôsobenia} % (fold)
\label{sub:mo_nosti_prisp_sobenia}

Existuje mnoho možností, na základe ktorých môžme prispôsobovať webové aplikácie jednotlivým zariadeniam. Hlavným prvkom adaptácie je veľkosť a rozlíšenie displeja cieľového zariadenia. Ďalšími ale nemej dôležitými sú spôsoby ovládania zariadenia, jeho pripojenie na internet alebo samotná platforma.

\subsubsection{Veľkosť a rozlíšenie} % (fold)
\label{ssub:ve_kos_a_rozl_enie}

Jednou zo základných možností prispôsobenia webových aplikácii je adaptácia na základe zobrazovacieho displeja zariadenia. Displeje môžu mať rozličné fyzické veľkosti ale aj rozlíšenia.

Časy s ,,rovnakým'' displejom na všetkých zariadeniach a podporou jednotného statického rozlíšenia na webových stránkach sú už dávno preč. S príchodom nových malých mobilných zariadení sa potreba prispôsobenia ešte viac umocnila. Veľkosti a rozlíšenia zariadení sa postupne začínali prelínať, dokonca v súčasnosti sa na trh uvádzajú zariadenia s väčším rozlíšením ako majú monitory stolových počítačov.


\begin{figure}[H]
	\centering
	\includegraphics[width=0.75\textwidth]{img/tnav-devices.jpg}
	\caption[Porovnanie zariadení vzhľadom na veľkosť displeja]{
		Porovnanie zariadení vzhľadom na veľkosť displeja \cite{navigation}.\\
		Prevzaté z http://www.lukew.com/ff/entry.asp?1649}
	\label{fig: tnavmobile}
\end{figure}

Keďže sa začínajú vyskytovať rovnaké rozlíšenia displaja na rozlične veľkých zariadeniach alebo opačne, je potrebné medzi zariadeniami rozlíšovať aj inými spôsobmi. Je potrebné správne poruzumieť jednotkám ,,pixel'' a ,,viewport''.

Na rozdiel od pixelu definovaného W3C pomocou pozorovacieho uhlu a vzdlialenosti \cite{w3cpixel} existujú rôzne iné bežne používané jednotky ,,CSS pixel'', ,,device pixel'' a ,,density-independent pixel'' \cite{pixelnotpixel}. CSS pixel je abstrakný, môže sa zvyšovať alebo zniživať, používa sa bežne v kóde na definovanie rozmerov elementov.  Device pixel je fyzický pixel nachádzajúci sa na zariadení. Pretože zariadenia majú stále viac fyzických pixelov a tým aj ich väčsiu hustotu, zaviedol sa pojem density-independent pixel. Ten je opäť abstraktný a predstavuje počet CSS pixelov optimálnych na prezeranie obsahu. Pokiaľ by nebol zavedený, tak zariadenia s veľkou hustotou pixelov by sa nedali použiť na bežné prezeranie obsahu, nakoľku pixely sú veľmi malé a zobrazený text alebo elementy by tak boli nečitaľné.

Viewport je celkové miesto potrebné na zobrazenie webovej stránky. Na mobilných zariadeniach je situácia komplikovanejšia, pretože stále existuje množstvo stránok, ktoré nie sú na ne optimalizované. Preto ho výrobcovia mobilných prehliadačov rozdelili na ,,layout viewport'' a ,,visual viewport''. \cite{pixelnotpixel} Layout viewport je pre rozmiestnenie elementov celej stránky a visual viewport je definovaný pre elemty po priblížení stránky tak, že sa nezmestila na displej zariadenia.

% subsubsection ve_kos_a_rozl_enie (end)

\subsubsection{Interakčné prostriedky} % (fold)
\label{ssub:interak_n_prostriedky}

Postupom času začína upadať používanie klasických stolových počítačov ovládaných pomocou klávesnice a myši a presadzujú sa nové druhy mobilných zariadení so vstupným interfejsom v podobe množstva senzorov, ale hlavne s dotykovou plochou. Tá sa ako ovládací prostriedok začína presadzovať okrem mobilných zariadení aj v notebookoch. Pri dizajne aplikácie je preto potrebné myslieť aj na takýchto používateľov. Okrem nich však existujú aj iné možnosti ovládania ako sú napríklad sledovanie pohybu pomocou web kamery, natočanie a posúvanie zariadenia získané pomocou gyroskupu a accelerometra či počúvanie hlasových povelov zaznamenaných cez mikrofón. Je tak potrebné brať do úvahy aj ďalšie možnosti.

% \newpage

\paragraph{Dotyk} % (fold)

Dôležitým prvkom v prípade mobilných zariadení je možnosť ovládania aplikácii jednou rukou. Mobilné zariadenia sú využívané takmer pri každej príležitesti či vo vonkajšom alebo vnútornom prostredí, preto je potrebné správne prispôsobiť vzhľad a rozmiestnenie ovládacích prvkov. Pre ne platí, že najjednoduchšie dosiahnuteľné časti sú v spodnej strane zariadenia a s postupom k hornému okraju možnosť dosiahnutia klesá \cite{mobilebooktouch}.

\begin{figure}[H]
        \centering
        \begin{subfigure}[b]{0.7\textwidth}
                \centering
                \includegraphics[width=\textwidth]{img/tnav-touch-phones.png}
        \end{subfigure}%
         %add desired spacing between images, e. g. ~, \quad, \qquad etc.
          %(or a blank line to force the subfigure onto a new line)
        \begin{subfigure}[b]{0.2\textwidth}
                \centering
                \includegraphics[width=\textwidth]{img/tnav-touch-phones2.png}
        \end{subfigure}

        \caption[Schopnosť ovládania mobilných telefónov]{Schopnosť ovládania mobilných telefónov \cite{navigation}.\\
		Prevzaté z http://www.lukew.com/ff/entry.asp?1649}
		\label{fig:tnavphones}
\end{figure}

Väčšie mobilné zariadenia alebo tablety už nie je možné pohodlne udržať v jednej ruke a tak je dôležité prispôsobiť ovládanie na dve ruky. Tablety sú držané v dvoch rukách za hranu a tak najlepšie dosiahnuteľné miesta sú na jeho okrajoch \cite{mobilebooktouch}.

\begin{figure}[H]
        \centering
        \begin{subfigure}[b]{0.6\textwidth}
                \centering
                \includegraphics[width=\textwidth]{img/tnav-touch-tablets.png}
        \end{subfigure}%
         %add desired spacing between images, e. g. ~, \quad, \qquad etc.
          %(or a blank line to force the subfigure onto a new line)
        \begin{subfigure}[b]{0.4\textwidth}
                \centering
                \includegraphics[width=\textwidth]{img/tnav-touch-tablets2.png}
        \end{subfigure}

        \caption[Schopnosť ovládania tabletov]{Schopnosť ovládania tabletov \cite{navigation}.\\
		Prevzaté z http://www.lukew.com/ff/entry.asp?1649}
		\label{fig:tnavtablets2}
\end{figure}

Podobné výsledky \cite{mobilebooktouch} sú aj pri novej kategórii zariadení, tzv. hybridných notebookov, ktoré okrem klávesnice obsahujú aj dotykový displej.

\begin{figure}[H]
	\centering
	\includegraphics[width=0.6\textwidth]{img/tnav-touch-laptops.png}
	\caption[Schopnosť ovládania hybridných počítačov]{
		Schopnosť ovládania hybridných počítačov \cite{navigation}.\\
		Prevzaté z http://www.lukew.com/ff/entry.asp?1649}
	\label{fig: tnavlaptops}
\end{figure}



\paragraph{Reč} % (fold)

Možnosť rozpoznania reči vo webovej aplikácii je v súčasnosti experimentálna novinka a jej štandartizacia je zatiaľ veľmi ďaleko. Používateľ samozrejme musí explicitne povoliť použitie mikrofónu pre webovú aplikáciu. Špecifikácia je zatiaľ len vo forme návrhu a nie je ani zaradená do W3C štandartu HTML5 \cite{webspeechapi}. Podpora v prehliadočoch s enginom Blink sa však už nachádza od začiatku roku 2013 a základná funkčnosť bola odprezentovaná v rámci konferencie Google IO 2013. Rozpoznanie reči prebieha vzdialene na serveroch patriacich Googlu a zatiaľ neexistuje možnosť rozpoznania lokálne \cite{moreawesomeweb}.

S rozpoznaním reči je spojená aj jeho syntéza, alebo preklad textu na reč. Tá je na tom v súčasnosti čo sa týka podpory a implementácie v prehliadačoch ešte horšie. Experimentálna funkčnosť existuje len v posledných buildoch prehliadočov. Našťastie sa táto funkcionala dá čiastočne nahradiť volaniami vzdialených webových služieb ako je napríklad ,,Google Translate''. Takáto syntéza však už neprebieha na zariadení, ale len sa zo serveru príjme zvuková nahrávka, ktorá sa následne prehrá.

\paragraph{Kamera} % (fold)

Moderné webové prehliadače umožňujú získať prístup aj k video streamu z web kamery používateľa. Na prístup je rovnako potrebné povolenie od používateľa. Tento video stream prebieha vo zvolenej frekvencii a sa dá odchytiť a následne uložiť do elementu canvas, kde už sa k jednotlivým vzorkám pristupuje ako k obrázku. Je možné odfiltrovať okolie a zachovať len podstatné informácie na základe ktorých sa získa pohyb, respektíve gesto od používateľa.


\paragraph{Accelerometer a Gyroskop} % (fold)

Medzi dlhodobo podporované senzory najmä v mobilných zariadeniach patria accelerometer a gyroskop. Je k nim umožnený prístup priamo z webovej aplikácie aj bez priameho povolenia od používateľa.

Accelerometer slúží na získanie zrýchlenia zariadenia v osiach x, y a z, gyroskop na získanie uhlového zrýchlenia okolo týchto osí. Ich rozdiel je znázornený na nasledujúcom obrázku:

\begin{figure}[H]
  \centering
  \includegraphics[width=0.8\textwidth]{img/accvsgyro.png}
  \caption[Accelerometor a gyroskop]{
    Zaznamenávané údaje pomocou accelerometora a gyroskopu}
  \label{fig: tnavlaptops}
\end{figure}

% subsubsection interak_n_prostriedky (end)


\subsubsection{Pripojenie} % (fold)
\label{ssub:pripojenie}

Spoločnou vlastnosťou webových aplikácii je, že pristupujú k rôznym zdrojom, ktoré sa môžu nachádzať okrem lokálneho úložiska aj na serverom, pomocou internetu. Spôsoby prenosu dát sú rôzne, od pevného pripojenie cez bezdrôtové až po mobilné, a každé z nich má iné vlastnosti.

V poslednom období sa spolu s mobilnými zariadeniami rozširuje aj používanie mobilného internetu. Keďže našim cieľom je, aby sa webová stránka načítala používateľovi čo najrýchlejšie, respektíve ak chceme aby sa používateľ na našu stránku prišiel a príchod si nerozmyslel pri jej dlhom načítavní, tak musíme šetriť množstvom prenášaných dát. Takéto šetrenie dát zároveň šetrí aj peňaženky používateľov \cite{performance}, hlavne pokiaľ sa jedná o roamingové dáta v zahraničí.

Načítanie webovej stránky sa skladá z viacerých fáz. Okrem samotnej konektivity používateľa aj spracovanie požiadavky serverom a následne vykonanie akcie v prehliadači. To je jediná fáza, ktorú môžeme ovplyvniť.

\begin{figure}[H]
	\centering
	\includegraphics[width=1.0\textwidth]{img/w3c-timing-overview.png}
	\caption[Fázy spracovania požiadavky na server]{
		Fázy spracovania požiadavky na server \cite{timing}.\\
		Prevzaté z http://www.w3.org/TR/navigation-timing/}
	\label{fig: timing}
\end{figure}

Tieto fázy môžme aj priamo merať pomocou interfacu \url{performance.timing}, ktorý ich zobrazuje priamo v podobe času \cite{1000ms, performancebrowsernetworking}. Vďaka tomu máme dokonalejší prehľad o používateľovom pripojení a vieme mu tak prispôsobiť jednotlivé komponenty stránky. W3C špecifikácia je vo fáze ,,Recommendation'' \cite{timing}, ale stále hlavnou nevýhodou je chýbajúca podpora v niektorých prehliadačoch. Čiastočnou náhradou je meranie rozdielov dvoch časov, ale tak získame dáta len zo spracovania požiadavky na strane klienta.

\begin{figure}[H]
	\centering
	\includegraphics[width=1.0\textwidth]{img/1000ms.png}
	\caption[Čas potrebný na spracovanie požiadavky na server]{
		Čas potrebný na spracovanie požiadavky na server.}
	\label{fig: 1000ms}
\end{figure}

Každá požiadavka na server niečo stojí. Ideálny prípad je taký, že medzi zariadením a serverom sa neprenášajú žiadne dáta a všetky prístupy ku zdrojom sa riešia len z lokálneho úložiska. V prípade internetových aplikácii to však väčšinou nie je úplne možné, pretože používateľ chce pristupovať k čo najčerstvejším dátam. Cieľom je však čo najviac limitovať požiadavky na server.

To, či je vôbec používateľ pripojený na internet vieme zistiť pomocou interfacu \url{navigator.onLine}, ktorý vráti hodnotu ,,true'' alebo ,,false'' a takisto môžme počúvať na zmeny pripojenia vďaka ,,event listenerom'' na \url{window.online} a \url{window.offline}.

Rýchlosť, akou je používateľ pripojený na internet, je dostupná v objekte \url{navigator.connection} pod atribútom ,,bandwidth'' charakterizujúcej pripojenie v MB/s \cite{network}. V prípade zmeny rýchlosti je rovnako vyvolaný ,,event''. Nevýhodou je zatiaľ slabá podpora zo strany prehliadačov.

% subsubsection pripojenie (end)

\newpage
\subsubsection{Platforma} % (fold)
\label{ssub:platforma}

Rozhodovanie sa na základe platformy je taktiež veľmi dôležité. Umožňuje nám jednak zjednošiť dizajn a zmenšiť počet potrebných komponentov na webovej stránke, ale taktiež vytvárať cielenú reklamu. Rozlišovanie prebieha na základe pola ,,user agent'' špecifickom pre každý prehliadač, respektíve operačný systém.

V prípade zisťovania podpory jednotlivých vlastností je však lepšie priamo zisťovať podporu komponentu zo strany prehliadača ako zisťovaním a porovnaním s platformou. Nemusíme si tak udržiavať databázu a neustále ju aktualizovať. Takéto riešenie je preto z pohľadu vývoja lepšie pre budúcnosť.

% subsubsection platforma (end)

% subsection mo_nosti_prisp_sobenia (end)


\subsection{Adaptačné techniky \footnote{Tejto kapitole som sa už z časti venoval vo svojej bakalárskej práci Tvorba bohatých internetových aplikácií pre mobilné zariadenia \cite{ja}. V tomto dokumente sa nachádza rozšírená verzia doplnená o novo vzniknuté techniky adaptácie.}} % (fold)
\label{sub:adapta_n_techniky}

S príchodom prvých mobilných zariadení existoval rozdiel medzi mobilným webom a webom pre desktopy a tak bolo pomocou servera jednoduché zistiť, aká verzia sa má zariadeniu zobraziť.

Pretože dnes už existuje mnoho zariadení od mobilných cez tablety až po klasické počítače a vzájomne sa prelínajú, je potrebné zabezpečiť, aby sa webový obsah zobrazoval správne na každom z nich.

\begin{fancybox}
\textit{,,There is no Mobile Web. There is only The Web, which we view in different ways. There is also no Desktop Web. Or Tablet Web. Thank you.'' Stephen Hay} \cite{noMobileWeb}
\end{fancybox}

Tento výrok bol vyslovený už pred niekoľkými rokmi a v súčasnosti pri prelínaní rôznych zariadení sa stále viac potvrdzuje. Pre vývoj webovej stránky alebo aplikácie existuje viacero spôsobov \cite{mobiforge}, každý má svoje výhody a nevýhody. Správnosť výberu konkrétnej metódy záleži od toho, či ideme upravovať už existujúcu webovú verziu na rôzne zariadenia alebo či vytvárame novú aplikáciu a v neposlednom rade aj od vynaloženého úsilia či financií.


\subsubsection{Responsive design} % (fold)
\label{ssub:responsive_design}

Pojem ,,Responsive design'' bol pôvodne súbor pravidiel tvorby dizajnu pre rôzne rozlíšenia, ktoré definoval Ethan Marcotte v článku Responsive design v roku 2010. Všetkým zariadeniam je posielaný rovnaký HTML a javascript, rozdiel je len v designe. Design sa zakladá na používaní flexibelného vzhľadu stránky, ktorý sa prispôsoboval rôznym zariadeniam, flexibilných obrázkoch, ktoré sa prispôsobujú vzhľadu a CSS media queries. \cite{responsive, mediaqueries} Až neskôr bol označený ako metóda na dosiahnutie výsledku.

Tvorba vzhľadu pomocou Responsive design znamená používanie hodnôt v percentuálnom pomere namiesto statických hodnôt, obrázky sú prepojené s elementom stránky, majú nastavené jeho plné rozmery a automaticky sa prispôsobujú jeho zmenám. Vytvára sa stránka pre väčšie rozlíšenie a pomocou CSS media queries sa môžu aplikovať rôzne pravidlá pre jednotlivé elementy na základe rozlíšenia zariadenia, jeho orientácie či pomeru strán. Pri použití takéhoto prístupu však nastávajú problémy na mobilných zariadeniach s menším rozlíšením displeja a na väčších zariadeniach ako sú televízory.\\


% \begin{center}
% 	\includegraphics[width=1.00\textwidth]{img/responsive.png}
% 	\captionof{figure}{Stránka http://www.londonandpartners.com/ na rôznych zariadeniach}
% 	\label{fig: responsiveImg}
% \end{center}

\paragraph{Výhody:}
\begin{itemize}
	\item dobrá metóda na dosiahnutie nezávislosti zobrazenia obsahu pri rôznom rozlíšení zariadení, stránka vyzerá inak v mobilnom zariadení ako v tablete alebo stolnom počítači
	\item rýchly vývoj aplikácie
\end{itemize}

\paragraph{Nevýhody:}
\begin{itemize}
	\item neumožňuje prispôsebenie obsahu, ale len jeho vzhľad
	\item mobilné zariadenie sťahuje plnú veľkosť obrázka, ale vidí ho v menšom rozlíšení
	\item problémy pri zariadeniach s nižším a väčším rozlíšením
\end{itemize}

% subsubsection responsive_design (end)


\subsubsection{Mobile First} % (fold)
\label{ssub:mobile_first_responsive_design}

Problémy pri správnom zobrazení stránky na zariadeniach s nižším rozlíšením podnietili vznik novej metódy dizajnu - ,,Mobile First''. Základ tvorí metóda Responsive design, ale pôvodný návrh stránky sa nerobí pre desktopovú verziu ale na malé rozlíšenie. Až pomocou media queries sa pridávajú elementy pre väčšie rozlíšenie. Takto sa pokryju všetky zariadenia od najmänších po najväčšie a webová stránka je pripravená na zariadenia, ktoré vzniknú aj v budúcnosti.

Technika Mobile First však okrem samotného dizajnu zahŕňa aj optimalizáciu webu pre používateľa či už z pohľadu UX alebo výkonu \cite{mobilefirst}.

\paragraph{Výhody:}
\begin{itemize}
	\item dosiahnutie nezávislosti zobrazenia obsahu pri rôznom rozlíšení zariadení
	\item podporuje všetky zariadenia od najmenších po najväčšie, stránka je pripravená aj na ,,zariadenia budúcnosti''
\end{itemize}

\paragraph{Nevýhody:}
\begin{itemize}
	\item stále neumožňuje prispôsebenie obsahu, ale len jeho vzhľad
	\item dizajn stránky musí byť vytvorený od základu, čo však môže byť aj východa
\end{itemize}

% subsubsection mobile_first_responsive_design (end)

\subsubsection{Progressive enhancement} % (fold)
\label{ssub:progressive_enhancement}

V poslednom období sa stáva veľmi popolárnou metódou Progressive Enhancement. Táto metóda je založená na princípe posielania rovnakého html, javascriptu a iných zdrojových súborov všetkým zariadeniam. Následné vykonávanie aplikácie sa presúva zo strany servera a prebieha na klientovi pomocou javascriptu, kde je už možné presne špecifikovať, čo sa má kedy a ako vykonávať. Používateľovi sa sprístupňujú pokročilejšie vlastností aplikácie len ak ich podporuje používaný webový prehliadač, respektíve zariadenie. Taktiež je možné načítavať objekty zo servera len vtedy, keď sú potrebné, a tým sa zabraňuje zbytočným prenosom dát. Pri spojení tejto metódy s HTML5 je dokonca možné tvoriť offline klientské aplikácie.

Pokiaľ chceme pokryť celé spektrum zariadení tak je implementácia pomocou tejto metódy náročnejšia. V spojení s metódou reseponsive design je to najlepšie možné riešenie na prispôsobenie si obsahu a vzhľadu aplikácie.

\paragraph{Výhody:}
\begin{itemize}
	\item úplne prispôsobenie obsahu a vzhľadu, minimálne alebo žiadne dátove prenosy
\end{itemize}

\paragraph{Nevýhody:}
\begin{itemize}
	\item náročnejšia implementácia, 
	\item rýchlosť vykonávania aplikácie závisi od výkonu zariadenia
\end{itemize}

% subsubsection progressive_enhancement (end)

\subsubsection{Graceful degradation} % (fold)
\label{ssub:graceful_degradation}

Opak metódy Progressive enhancement sa nazýva Graceful degradation. Používateľovi sa predáva plne funkčná aplikácia nadizajnovaná na najlepšie zariadenia, kde sa na jednotlivé pokročilejšie funkcie postupne vypínajú vzhľadom na použité zariadenie. Používa sa často na upravenie súčasnej nasadenej verzie na potreby mobilných zariadení.

\begin{figure}[H]
	\centering
	\includegraphics[width=0.8\textwidth]{img/PEvsGD.jpg}
	\caption[Progressive enhancement vs Graceful degradation]{
		Progressive enhancement vs Graceful degradation \cite{adaptivesxsw}.\\
		Prevzaté z http://bradfrostweb.com/blog/web/mobile-first-responsive-web-design/}
	\label{fig: gd}
\end{figure}

\paragraph{Výhody:}
\begin{itemize}
	\item prispôsobenie obsahu a vzhľadu na jednoducšie zariadenia
	\item zachovanie súčasnej verzie webovej stránky
\end{itemize}

\paragraph{Nevýhody:}
\begin{itemize}
	\item nezohľadňuje budúce zariadenia, už v súčasnosti majú niektoré tablety kvalitnejšie displeje ako stolové počítače 
\end{itemize}

% subsubsection graceful_degradation (end)


\subsubsection{Server-side Adaptation} % (fold)
\label{ssub:ress_}

Metóda Server-side Adaptation je používaná väčšinou webových stránok na detekovanie mobilného zariadenia a v súčasnosti patrí už medzi historické techniky. Pri prístupe na stránku sa pomocou servera detekuje zariadenie a je mu ponuknutá vhodná verzia stránky, väčšinou dochádza k presmerovaniu (na mobilnú verziu). Celá logika aplikácie sa nachádza na serveri. Stránka môže byť presne vytvorená pre dané zariadenie, takže nenastávajú problémy pri dizajne, je však potrebné mať na serveri nainštalovanú knižnicu na jeho detekovanie\footnote{napríklad DeviceAtlas alebo WURFL založené na detekovaní pola user agent}. Detekcia zariadenia je pri priamej návšteve stránky cez prehliadač úspešná, k problémom však prichádza pri návšteve stránky cez iných klientov. Taktiež je dôležité mať databázu zariadení neustále aktualizovanú.

\paragraph{Výhody:}
\begin{itemize}
	\item zobrazenie vhodnej stránky pre zariadenie, nesťahuje sa nepotrebný obsah
\end{itemize}

\paragraph{Nevýhody:}
\begin{itemize}
	\item potreba knižníc na detekovanie zariadenia, ktorú je nutnú neustále aktualizovať
	\item dlhšie načítavanie stránky na pomalšom internetovom pripojení, pretože dochádza k presmerovaniu
\end{itemize}

% subsubsection ress_ (end)

\subsubsection{RESS (Responsive Design + Server Side Components)} % (fold)
\label{ssub:ress_responsive_design_server_side_components_}

Kombináciou techník Responsive Design a Server-side Adaptation vznikla novšia technika adaptácie. Pre každé zariadenie sa na serveri dynamicky vygenerujú pre neho špecifické časti webovej aplikácie, ktoré sa mu následne zobrazia a upravia pomocou responsive dizajnu.

\paragraph{Výhody:}
\begin{itemize}
	\item jednoduchšia udržiaveteľnosť, neexistujú rôzne verzie stránky ale len jedna
\end{itemize}

\paragraph{Nevýhody:}
\begin{itemize}
	\item potreba nainštalovaných knižníc na detekovanie zariadenia
	\item náročnejšia implementácia na strane servera
\end{itemize}

% subsubsection ress_responsive_design_server_side_components_ (end)

% subsection adapt_cia (end)

\newpage
\section{Nástroj na overenie} % (fold)
\label{sec:n_stroj_na_overenie}
Ako nástroj na overenie sa vytvorila knižnica s názvom Angular-Adaptive \footnote{Angular-Adaptive \url{http://angular-adaptive.github.io/} je sprievodca vo svete adaptívneho web dizajnu.} do populárneho JavaScriptového frameworku AngularJS \footnote{AngularJS \url{http://angularjs.org/} je populárny JavaScriptový framework, ktorý rozširuje HTML o nové možnosti pre webové aplikácie.}, kde sa následne skúmali možnosti adaptácie komponentov pomocou metódy progressive enhancement. Hlavným dôvodom výberu AngularJS bola možnosť vytvárania modulov, ktoré sa dajú následne jednoducho použivať vo webových projektoch. Takáto podpora vytvárania externých modulov bude už o pár rokov natívna v prehliadačoch vďaka webovým komponentom \cite{webcomponents}. Samotná distribúcia modulov prebieha pomocou balíčkovaciemu nástroja Bower \footnote{Bower \url{http://bower.io/} je balíčkovací manažér pre web.}.

\begin{figure}[H]
  \centering
  \includegraphics[width=1.0\textwidth]{img/angularadaptive.png}
  \caption[Webová stránka vytvoreného projektu angular-adaptive]{
    Webová stránka vytvoreného projektu angular-adaptive}
  \label{fig: angularadaptive}
\end{figure}

\subsection{Alternatívne spôsoby ovládania} % (fold)
\label{sub:alternat_vne_sp_soby_ovl_dania}

Príchod mobilných zariadení priniesol a spopularizoval nový spôsob interakcie - ovládanie zariadenia pomocou dotyku. Ten však vždy nemusí byť ten najvýhodnejší najmä v prostredí webových aplikácií. Najmä mobilné zariadenia sú vybavené gyroskopom ktorý môže uľahčiť ovládanie aplikácie len vďaka nakláňaniu zariadenia do strán. Podobne môže reagovať zariadenie sledovaním okolia vďaka vstupu z webovej kamery, ktorá je vstavané v množstve zariadení. Typickým senzorom v zariadeniach je mikrofón umožňujúci zaznamenať zvuk, z ktorého sa analyzujú rečové povely a tie vedia zefektívniť ovládanie aplikácie.

\subsubsection{Reč} % (fold)
\label{ssub:re_}

Ovládanie mobilných zariadení pomocou rečových povelov sa v posledných rokoch teší narastajúcej popularite. Rozpoznávanie je ale špecifické pre jednotlivé mobilné platformy. Javascriptové Web Speech API však umožňuje pridanie rozpoznania reči aj do webovej aplikácie. Cieľom je umožniť ovládanie aplikácie zadávaním statických povelov, ale aj ovládať dynamicky vytvárané entity. Na ich základe bol vytvorený modul ktorý umožňuje pridať kompletné ovládanie webovej aplikácie. Používateľ všal musí povoliť prístup aplikácie k mikrofónu. Po jeho potvrdení už prebieha kontinuálne rozpoznávanie reči.

Po pridaní modulu do webovej aplikácie je potrebné si ho nakonfigurovať. Musia sa vytvoriť úlohy, ktoré definujú správanie aplikácie a prípadne aj reč, v ktorej prebieha rozpoznávanie. Implementované rozpoznávanie reči v prehliadači Google Chrome podporuje 32 rôznych jazykov a ďalšie prízvuky vrátane slovenčiny.

Rozpoznanie úloh prebieha na základe regulárneho výrazu. Keď sa aktuálne nastavený jazyk rozpoznávania zhoduje s jazykom v úlohe a rozpoznaný text spĺňa podmienku regulárneho výrazu, tak sa zavolá asociovaná callback funkcia. Tá v parametri obsahuje rozpoznaný text, s ktorým sa dá ďalej pracovať. 

Príklad konfigurácie na pridanie novej úlohy má nasledujúci tvar:

\begin{verbatim}
'someTask': {
  'regex': /^do .+/gi,
  'lang': 'en-US',
  'call': function(utterance){
    // DO SOMETHING
  }
}
\end{verbatim}

Konfiguráciu je možné zavolať globálne v rámci aplikácie alebo dynamicky vďaka podpore regulárneho výrazu pre dynamicky vytvárané elementy. Napríklad ak sa v aplikácii pracuje so zoznamom úloh, tak globálne je možné obsluhovať pridanie novej úlohy a dynamicky vytvárané elementy môžu mať vlastnú obsluhu, ktorá počúva na ich meno. Diagram navrhnutého algoritmu ovládania webovej aplikácie pomoc reči sa nachádza na nasledujúcom obrázku:

\begin{figure}[H]
  \centering
  \includegraphics[width=0.6\textwidth]{diagram/speech.png}
  \caption[Algoritmus ovládania aplikácie pomocou reči]{
    Algoritmus ovládania aplikácie pomocou reči}
  \label{fig: diaspeech}
\end{figure}

% subsubsection re_ (end)


\subsubsection{Pohyb} % (fold)
\label{ssub:pohyb}

Ovládanie webovej aplikácie pohybom prebieha vďaka webovej kamere. Cieľom je rozpoznať základné smery pohybov používateľa a umožniť na ne namapovať funkcionalitu aplikácie. Zachytený video stream sa sníma s frekvenciou 60 Hz. Každý jeden snímok je získaný z kamery a vykreslený do canvasu. Je tvorený množinou pixelov s rgba hodnotami a ten sa následne upraví pomocou hsv filtra aplikovaného na všetky pixely aby sa detekovali oblasti s kožou. Upravený snímok je tvorený dvomi farbami: tmavou znamenajúcou kožu a bielou pre ostatné oblasti.

Následne prebieha ďalšie upravovanie pri ktorom sa detekuju hrany. Ich detekcia prebieha na základe rozdielov farebnosti pixelov posledných dvoch snímkov s aplikovaným filtrom kože. Opäť vznikne snímok tvorený dvomi farbami, ale tentokrát tmavá farba sa nachádza len na hranách a biela na ostatných častiach obrázku.

Keď je obrázok v už takomto stave tak je možné začať rozpoznávať predvádzané gesto. Spočíta sa celkový počet zmenených bodov a pomer zmenených bodov v osiach x a y obrázka. Postupným hromadením snímkov a spočítavaním všetkých zmenených bodov sa určí celkový smer pohybu. Celkovo sa rozpoznávajú štyri základné gestá, kde po úspešnom rozpoznaní sa vykonajú priradené funkcie:

\begin{itemize}
  \item onSwipeLeft - pohyb zprava vľavo
  \item onSwipeRight - pohyb zľava vpravo
  \item onSwipeUp - pohyb zdola hore
  \item onSwipeDown - pohyb zhora dole 
\end{itemize}

Citlivosť aplikovania filtrov a rozpoznávania gesta je samozrejme možné konfigurovať nastavením tresholdu a úpravou hsv filtra. Vytvorený modul okrem toho podporuje aj ďalšie udalosti na ktoré je možné viazať funkcie akými sú začiatok a koniec snímania pretože na prístup k web kamere je potrebné explicitné povolenie od používateľa.  Diagram navrhnutého algoritmu ovládania webovej aplikácie pomoc pohybu sa nachádza na nasledujúcom obrázku:

\begin{figure}[H]
  \centering
  \includegraphics[width=0.8\textwidth]{diagram/motion.png}
  \caption[Algoritmus ovládania aplikácie pomocou pohybu]{
    Algoritmus ovládania aplikácie pomocou pohybu}
  \label{fig: diamotion}
\end{figure}

\paragraph{Vizualizácia} % (fold)
\label{par:zobrazenie_n_h_adu}

V aplikácii je možné pridať aj zobrazenie aktuálneho náhľadu z webovej kamery a to použitím vytvorenej direktívy pridaním atribútu ,,adaptive-motion'' ku canvas elementu. Atribút môže nadobúdať nasledujúce hodnoty:

\begin{itemize}
  \item video - zobrazí aktuálny náhľad z web kamery

\begin{figure}[H]
  \centering
  \includegraphics[width=0.3\textwidth]{img/motion/video.png}
  \caption[Vizualizácia videa zachzteného web kamerou]{
    Vizualizácia videa zachzteného web kamerou}
  \label{fig: motion-video}
\end{figure}

  \item skin - zobrazí video stream po aplikovaní filtra rozpoznania kože

\begin{figure}[H]
  \centering
  \includegraphics[width=0.3\textwidth]{img/motion/skin.png}
  \caption[Vizualizácia videa po aplikovaní filtra detekcie kože]{
    Vizualizácia videa po aplikovaní filtra detekcie kože}
  \label{fig: motion-skin}
\end{figure}

  \item edge - zobrazí video stream po aplikovaní filtra rozpoznania hrán

\begin{figure}[H]
  \centering
  \includegraphics[width=0.3\textwidth]{img/motion/edges.png}
  \caption[Vizualizácia videa po aplikovaní filtra detekcie hrán]{
    Vizualizácia videa po aplikovaní filtra detekcie hrán}
  \label{fig: motion-edges}
\end{figure}

\end{itemize}


% paragraph zobrazenie_n_h_adu (end)

% subsubsection pohyb (end)

\subsubsection{Gyroskop} % (fold)
\label{ssub:gyroskop}

Jednou z možností ovládania webovej aplikácie je použitie gyroskopu. Na tento prípad som vytvoril modul pomocou ktorého je možné v aplikácií skrolovať len pomocou natáčania zariadenia do strán bez ďalšej potrebnej interakcie používateľa. Modul sa inacializuje pridaním atribútu ,,adaptivescroll'' k elementu, v ktorom má prebiehať skrolovanie. Tým môže byť napríklad element ,,body'', ktorý zohľadňuje skrolovanie celej webovej stránky, alebo aj nejaký iný vnorený element. Následne stačí zavalať metódu so začatím sledovania polohy zariadenia, v ktorej sa môže upraviť ignorovaná hranica natočenia v uhloch a skrolovanie prebieha automaticky. V prípade potreby je toto sledovanie možné aj ukončiť.

Po začatí sledovania aktuálneho natočenia zariadenia sa uloží jeho počiatočná hodnota, ktorá sa používa ako relatívna hranica medzi stranami na ktoré prebieha skrolovanie. Postupným natáčaním zariadenia v smere hodinových ručičiek okolo osi beta od počiatočnej hodnoty k používateľovi sa vykonáva skrolovanie elemntu smerom dola, natáčaním na opačnú stranu od počiatočnej hodnoty prebieha skrolovanie hore.

Do úvahy pri zohľadňovaní rýchlosti skrolovania sa berie aj absolútna hodnota natočenia zariadenia od počiatočnej hodnoty zariadenia. Čím je táto hodnota väčšia, tým rýchlejšie prebieha skrolovanie. Samotný proces skrolovania prebieha vďaka číselnej zmene offsetu elementu. Plynulosť skrolovania je zaručená použitím animácii pomocou requestAnimationFrame. Tá prebieha s frekvenciou 60 Hz a je časovaná priamo prehliadačom. Tej však chýba podpora v straších verziách, v takom prípade musí byť použítý manuálny časovač aplikácie.

Okrem skrolovania modul obsahuje aj ďalšie metódy onalpha, onbeta a ongamma, ktorých návratová funkcia sa zavolá vždy po zmene rotácie zariadenia a je ich možné použiť na iné potreby v aplikácii. Funkcia obsahuje parameter absolútnu zmenu pozície v uhloch od počiatočnej k aktuálnej. Diagram navrhnutého algoritmu ovládania webovej aplikácie pomoc gyroskopu sa nachádza na nasledujúcom obrázku:

\begin{figure}[H]
  \centering
  \includegraphics[width=0.8\textwidth]{diagram/scroll.png}
  \caption[Algoritmus ovládania aplikácie pomocou gyroskopu]{
    Algoritmus ovládania aplikácie pomocou gyroskopu}
  \label{fig: diascroll}
\end{figure}

Po vytvorení modulu a testovaní na zariadeniach sa zistili problémy v skutočnej implementácii udalostí natáčania zariadenia v rôznych prehliadačoch. W3C špecifikácia orientácie zariadenia uvádza nasledovné:

\begin{itemize}
  \item os X je rovnobežná na rovinu zeme, kladné hodnoty má v smere na východ, záporné na západ
  \item os Y je rovnobežná na rovinu zeme, kladné hodnoty má v smere na sever, záporné na juh
  \item os Z je kolmá na rovinu zeme, kladné hodnoty má v smere od zeme, záporné smerom k zemi
\end{itemize}

Rotácia by mala byť popísaná pravidlom pravej ruky, teda kladné hodnoty rotácie sú v smere hodinových ručičiek okolo osí.

Nasledujúce tabuľky ukazujú, kde sa nachádzajú nulové body pre jednotlivé osi, aké hodnoty nadobúda rozsah rotácie a či je dodržané pravidlo pravej ruky.

Pravidlo pravej ruky (PPR) hovorí: 

\begin{fancybox}
\textit{,,Kladné hodnoty rotácie sa zvyšujú pri rotovaní okolo osi v smere hodinových ručičiek pri ukazovaní smerom na narastajúcu kladnú hodnotu smerovej osi.'' W3C \cite{deviceorientation}}
\end{fancybox}


Táto definícia je však protichodná s hodnotami kompasu. Pri rotácii okolo osi Z v smere rotácie kompasu sa hodnoty rotácie znižujú a nie narastajú. To je však spôsobené pohľadom do záporných hodnôt osi Z.


\paragraph{Alpha} % (fold)
\label{par:alpha}

Rotácia okolo osi Z nadobúda hodnoty rotácie alpha. V špecifikácii je nejasne definové aké by mali byť východiskové hodnoty, čo vo výsledku prináša rôzne implementácie rotácie v prehliadačoch. To spôsobuje zmätok nakoľko zo špecifikácie nie je jasné, na akú stranu by mal ukazovať nulový bod. Dá sa odpozorovať podľa pravidla pravej ruky, že 0 stupňov ukazuje smerom na sever, pretože implementovaných 90 stupňov ukazuje na východ.

Rozsah hodnôt bol testovaný pri držaní zariadenia v horizontálnej polohe a vycentrovaní smerom k nulovému bodu, keď hodnota rotácie bola 360 stupňov. Pozorované boli nasledujúce hodnoty:

\begin{table}[H]
  \begin{tabular}{ | l | l | l | l |}
  \hline
              & Nulový bod  & PPR   & Rozsah \\ \hline
  Reference   & Sever (0)   & A     & [0, 360] \\  
  iOS Chome   & Východ (90)   & A     & [0, 360] \\  
  iOS Safari  & Východ (90)   & A     & [0, 360] \\  
  Android & & & \\  
  Chrome      & Sever (0)   & A     & [0, 360] \\  
  Stock       & Západ (270)  & A     & [0, 360] \\  
  Firefox     & Sever (0)   & N     & [0, 360] \\
  \hline
  \end{tabular}
  \caption[Alpha rotácia gyroskopu]{Alpha rotácia gyroskopu}
\end{table}

% paragraph alpha (end)


\paragraph{Beta} % (fold)
\label{par:beta}

Rotácia okolo osi X nadobúda hodnoty rotácie beta. Špecifikácia definuje nulový bod ako horizontálnu polohu. Všetky testované prehliadače majú implementovaný smer osi rotácie správne, problémy však nastávajú pri rozsahu rotačných hodnôt.

Rozsah bol testovaný pri držaní zariadenia v horizontálnej polohe, čo je nulový bod. Zariadenie bolo potom otáčané okolo osi X o 90 stupňov tak, že displej zariadenia smeroval k pozorovateľovi. Potom nasledovalo otáčanie o ďalších 90 stupňov tak, že displej bol smerom k zemi. Na koniec nasledovala rotácia zvyšných 180 stupňov smerom do počiatočnej polohy, čím sa rotácia dokončila.


\begin{table}[H]
  \begin{tabular}{ | l | l | l | l | l |}
  \hline
              & Nulový bod    & PPR   & Rozsah         & Poznámky\\ \hline
  Reference   & Horizontálne   & A     & [0, -180|180] & \\  
  iOS Chome   & Horizontálne   & A     & [-90, 90]     & Plný rozsah rotácie nie je podporovaný. \\  
  iOS Safari  & Horizontálne   & A     & [-90, 90]     & Plný rozsah rotácie nie je podporovaný. \\  
  Android & & & & \\  
  Chrome      & Horizontálne   & A     & [-90, 90]     & Plný rozsah rotácie nie je podporovaný. \\  
  Stock       & Horizontálne   & A     & [-90, 90]     & Plný rozsah rotácie nie je podporovaný. \\  
  Firefox     & Horizontálne   & N     & [0, 180|-180] & Rotácie pokračuje naspäť na začiatok \\
  \hline
  \end{tabular}
  \caption[Beta rotácia gyroskopu]{Beta rotácia gyroskopu}
\end{table}

V iOS zariadeniach rovako ako v Android prehliadači a Chrome pre Android hodnoty rotácie od počiatku stúpajú správne. Ale keď rotácia nadobudne svoje maximum, čo je 90 a -90 stupňov, tak jej hodnoty začnú postupne klesať. Nie je tak podporovaný celý rozsah rotácie.

Vo Firefoxe pre Android je rotácia implementovaná opačným smerom a nie je dodržané špecifikované pravidlo pravej ruky. Je však podporovaný celý rozsah rotácie. Najskôr klesne k hodnote -180 stupňov, čo sa pri pokračujúcej rotácii zmení na kladnú hodnotu a potom klesá k nule.


% paragraph beta (end)

\paragraph{Gamma} % (fold)
\label{par:gamma}

Rotácia okolo osi Y nadobúda hodnoty rotácie gamma. Špecifikácia definuje nulový bod v horizontálnej polohe. Ale aj v tomto smere rotácie sú problémy s rozsahom rotácie, ale to je spôsobené najmä samotnou špecifikáciou zo strany W3C, nakoľko v nej sa nachádza rozsah len od -90 po 90 stupňov, čo je celkovo 180 a nie je tak pokryté celé otočenie.

Rozsah hodnôt bol testovaný pri držaní zariadenia v horizontálnej polohe a natočením na nulový bod. Zariadenie bolo potom otáčané okolo osi Y o 90 stupňov v smere hodinových ručičiek tak, že jeho displej smeroval vpravo. Potom nasledovalo otoče o ďalších 90 stupňov tak, že zariadenie smerovalo dolu. Na koniec sa dokočilo zostávajúcich 180 stupňov rotácie do pôvodného stavu.

\begin{table}[H]
  \begin{tabular}{ | l | l | l | l | l |}
  \hline
              & Nulový bod    & PPR   & Rozsah         & Poznámky\\ \hline
  Reference   & Horizontálne   & A     & [0, 90|-90]   & Zlá definícia od W3C \\  
  iOS Chome   & Horizontálne   & A     & [0, 180|-180] & Plný rozsah rotácie nie je podporovaný \\  
  iOS Safari  & Horizontálne   & A     & [0, 180|-180] & Plný rozsah rotácie nie je podporovaný. \\  
  Android & & & & \\  
  Chrome      & Horizontálne   & A     & [0, 270|-90]  & Upravený rozsah rotácie definuje natočenie \\  
  Stock       & Horizontálne   & A     & [0, 270|-90]  & Upravený rozsah rotácie definuje natočenie \\  
  Firefox     & Horizontálne   & N     & [0, -90|90]   & Rozsah rotácie pokračuje naspäť na začiatok \\
  \hline
  \end{tabular}
  \caption[Gamma rotácia gyroskopu]{Gamma rotácia gyroskopu}
\end{table}

Zlá definícia rotácie od W3C špecifikuje rotáciu len po hodnotu 90 stupňov z horizontálnej polohy, čo spôsobuje problémy pri väčšom natočení zariadenia.

V iOS rotácia nadobúda nulový bod v horizontálnej podobe s displejom smerujúcim hore. Otočením zariadenia okolo osi Y tak, že obrazovka bude smerovať smerom dole bude aktuálne gamma natočenie nadobúdať hodnotu 180, respektíve -180 stupňov. Ak sa rotácia uskutočňuje v smere hodinových ručičiek, tak hodnoty rotácie sa postupne zvyšujú v kladných číslach. Ak rotácia ide opačným smerom proti smeru hodinových ručičiek, tak hodnoty rotácie sa zmenšujú. Pri natočení zariadenia smerom na bok lavou strano hore je hodnota rotácia 90 stupňov, na opačnej strane pri smerovaní pravej hrany smerom hore je to -90 stupňov.

Chrome pre Android a vstavaný Android prehliadač majú spravné hodnoty rotácie od -90 po +90 stupňov. Ale po natočené zariadenia o viac ako 90 stupňov v smere hodinových ručičiek hodnota rotácie naďalej stúpa až po hodnotu 270 stupňov, čo je zároveň aj hodnota -90. Toto poskytuje možnosť presne vedieť veľkosť natočenia zariadenia.

Firefox má aj pri rotácii okolo osi Y podobný problém, a to je že hodnoty sú opačné a nedodržiava špecifikované pravidlo pravej ruky. Rozsah je správny, ale rotácie v smere hodinových ručičiek nadobúdajú záporné hodnoty a naopak.

% paragraph gamma (end)


\paragraph{Gyrocopter} % (fold)
\label{par:gyrocopter}

Na vyriešenie problémov pri vývoji webových aplikácii spôsobených rôznymi implementáciami rotácii v prehliadačoch som vytvoril rozšírenie Gyrocopter pre prehliadač Chrome \footnote{Dostupné na stiahnutie z Chrome Web Store \url{https://chrome.google.com/webstore/detail/gyrocopter/oooalfgemajfclliinfcdkifafmcfjop}}. Hlavným cieľom rozšírenia je umožniť zvoliť si simuláciu konkrétneho prehliadača a dynamicky meniť rotáciu zariadenia bez potreby testovania funkcionality na reálnom zariadení. Podobná funkcionalita sa už nachádza v developerských nástrojoch prehliadača Chrome, ale tam je možné len manuálne zadať hodnoty rotácie bez znalosti skutočných rozsahov a náhľadu natočenia zariadenia.

Toto rozšírenie umožňuje simulovať udalosti ktoré nastanú pri rotácii zariadenia, pričom používateľ môže natáčať zariadenie vo všetkých troch osiach. Rozšírenie je implementované pre vývojárov priamo do prostredia ,,Chrome Developer Tools''.

Nakoľko hodnoty rotácii sú implementované v prehliadačoch rôzne, tak používateľ má na výber zvoliť si podľa akej implementácie chce vyvolávať udalosti rotácie. Simulovanými prehliadačmi na výber sú:

\begin{itemize}
  \item W3C špecifikácia
  \item iOS Safari (Chrome používa web view)
  \item Android Chrome
  \item Android prehliadač
  \item Firefox pre Android
\end{itemize}

Samotné používateľské rozhranie rozšírenia obsahuje náhľad aktuálnej rotácie zariadenia na ktorom je možné zmeniť aktuálne simulovaný prehliadač. Pod ním sa nachádzajú uhly natočenia alpha, beta a gamma, ktoré je možné meniť. Tieto obsahujú pre jednotlivé prehliadače rozsahy a smer natočenia. Pri zmene rotácie na jednotlivých osiach sa zmena automaticky prejaví aj na náhľade zariadenia. 

Toto rozhranie navrhnutého rozšírenia je znázornené na nasledujúcom obrázku:

\begin{figure}[H]
  \centering
  \includegraphics[width=0.35\textwidth]{img/gyrocopter.png}
  \caption[Gyrocopter - gyroskop simulátor]{
    Gyrocopter - gyroskop simulátor}
  \label{fig: gyrocopter}
\end{figure}

Po výbere, respektíve zmene prehliadača, ktorý sa má simulovať, sa automaticky prepočítajú rozsahy pre jednotlivé uhly otáčania. Zároveň sa upravia na nich aj aktuálneho hodnoty natočenia bez toho, aby sa musela znázornená rotácia zariadenia meniť.


% paragraph gyrocopter (end)

% subsubsection gyroskop (end)

% subsection alternat_vne_sp_soby_ovl_dania (end)

% \newpage
\subsection{Komponenty} % (fold)
\label{sub:komponenty}

Pri adaptívnom dizajne sú dôležité aj webové komponenty s ktorými používateľ webovej aplikácie priamo interaguje. Tieto boli skúmané najmä z hľadiska optimalizácie na rôzne zariadenia a rýchlosti pripojenia.

\subsubsection{Video} % (fold)
\label{subsub:video}

Populárnym doplnkom súčasných webový stránok je priložené video, ktoré sa často nachádza na serveroch YouTube alebo Vimeo.

Problémom je, že pri vložení videa pomocou iframe alebo embed api sa uskotočňuje množstvo dopytov na servery a prenášajú sa zbytočné dáta aj keď používateľa video nezaujíma a vôbec si ho neprehrá. Okrem zbytočne prenášaných dát sa aj znižuje výkon, nakoľko určitý čas trvá spracovanie požiadaviek.

Nasledujúce dáta sa prenesú pri zobrazení stránky, na ktorú bolo vložené video zo servera YouTube pomocou dostupného iframe API:

\begin{figure}[H]
	\centering
	\includegraphics[width=1.0\textwidth]{img/youtube.png}
	\caption[Prenášané dáta pri požiadavke na video zo servera YouTube]{
		Prenášané dáta pri požiadavke na video zo servera YouTube}
	\label{fig: youtube}
\end{figure}

Celkovo sa vykoná 6 požiadaviek a prenesie sa 350 kB dát bez toho, aby používateľ spustil video. Podobná situácia sa opakuje aj pri požiadavke na video zo servera Vimeo pomocou iframe API.

\newpage
\begin{figure}[H]
	\centering
	\includegraphics[width=1.0\textwidth]{img/vimeo.png}
	\caption[Prenášané dáta pri požiadavke na video zo servera Vimeo]{
		Prenášané dáta pri požiadavke na video zo servera Vimeo}
	\label{fig: vimeo}
\end{figure}

V tomto prípade sa dokonca vykoná 15 požiadaviek a prenesie sa 120 kB dát. V prípade použitia javascriptového API s HTML 5 prehrávačom videa sa síce nestiahnú flashové komponenty, ale nahradia ich rovnakoveľké javascriptové súbory.

Lepšie riešenie je použiť podmienené načítavanie. Najskôr sa načíta len obrázok videa a pridajú sa jednoduché štylistické prvky aby vytvorený element pripomínal video súbor a až po kliknutí sa urobia dopyty na vzdialený server s automatickým prehratím videa. Výhodou takéhoto riešenia je zmenšenie veľkosti dopytov a s tým spojené šetrenie dát používateľov. 

Čo sa týka získania obrázkového náhľadu videa, tak YouTube túto možnosť priamo poskytuje a záleží len od identifikátora videa. K obrázku je tak možné pristúpiť priamo. Navyše po vykonaní požiadaviek na server po kliknutí na obrázok a následnom spustení videa pomocou api sa už daný onrázok nachádza v pamäti, tak sa nemusia robiť ďalšie požiadavky. 

Samotný proces automatického spustenia YouTube videa nie je úplne jednoduchý, nakoľko nastavenie hodnoty ,,autoplay'' pri vložení elementu iframe s videom na stránku nefunguje na mobilnom prehliadači Safari, kde je táto možnosť zakázaná. V takomto prípade by sa po kliknutí na obrázok videa načítal len element s videom a aby sa samotné video začalo prehrávať očakáva sa ďalšie kliknutie. Takéto správanie je neželené a je proti používateľskému zážitku. 

Preto bolo potrebné vymyslieť lepšie riešenie. To spočíva v načítaní scriptu s javascriptovým youtube API a následným vytvorením iframe elementu s videom až pomocou neho. Takto máme prístup k programátorskému ovládaniu prehravača videa a môžme ho spustiť keď potrebujeme, teda po kliknutí na obrázok s náhľadom videa. Takéto riešenie už fungujé aj na iOS s mobilným prehliadačom Safari.

Stále však máme problém pri staršej verzii iOS 5 a menej, tam nefunguje ani programátorske spustenie videa. V tejto verzii iOS sa však ešte distribuovala predinštalovaná aplikácia na prehrávanie Youtube videí, ktorá bola v neskorších verziách odstránená a je ju možné stiahnúť z iTunes. Výhodou je, že video môžme otvoriť priamo v nej pomocou youtube url schémy. Musíme však používaný prehliadač správne detekovať a následne sa rozhodnúť, aké prehrávanie zvolíme.

Na detekovanie funkcionality otvorenia youtube videa v aplikácii pomocou systému iOS a jej zisťovaním len z pola ,,user agent'' nie je správne, nakoľko by bolo potrebné získať úplne všetky verzie systému, ktorých je mnoho, a následne porovnať s verziou zariadenia.

Lepším riešením je vytvorenie testu funkčnosti prehrávania videa, ktoré je riešené pomocou nástroja Modernizr\footnote{Modernizr \url{http://modernizr.com/} je JavaScriptová knižnica, ktorá detekuje dostupnosť natívnej imlementácie nových technológií v prehliadači.} umožňujúcim detekciu základných HTML5 a CSS3 vlastností prehliadača a vytváranie vlastných testov. Následne sa na základe výsledku testu rozhodnemu akú akciu vykonať. Problémom je, že testy sa vykonávajú po načítaní stránky a nechceme používateľovi automaticky otvoriť natívnu aplikáciu alebo ho presmerovať na stránky youtube. Preto je zvolená iná varianta a detekuje sa podbora vlastnosti, ktorá bola pridaná až v novšej verzii iOS 6. Konkrétne sa jedná o podboru jednotiek ,,vh a vw'' (viewport height a viewport width) slúžiacich na nastavenie veľkosti DOM elementu. Detekcia či je zariadenie iOS vychádza z pola user agent, ale už sa nezisťuje jej verzia.

Výsledok je taký, že po kliknutí na obrázok s náhľadom videa a vyhodnotení testu prehrávania sa začne prehrávať v prehliadači alebo v natívnej aplikácii.

V prípade Vimea je situácia trochu komplikovanejšia pretože k náhľadovému obrázku sa priamo nevieme dostať a je potrebné urobiť jednu požiadavku na API, ktorá vráti informácie o videu. Vykonanie požiadavky síca nejaký čas trvá, ale aj tak je výsledok z pohľadu prenášaný údajov lepší ako robiť požiadavku priamo na samotné video.

S prehrávaním videa zo služby vimeo je situácia podobná, neexistuje však natívna aplikácia a po klinutí na náhľad videa je používateľ v starších verziách iOS presmerovaný na stránky vimea, v novších verziách iOS a v Androide sa mu automaticky prehrá.

% subsubsection video (end)


\subsubsection{Mapy} % (fold)
\label{subsub:mapy}

Mapy podobne ako videá vytvárajú nechcené dátové prenosy, dokonca ich ešte aj prevyšujú pretože sa neprenášajú len údaje o aktuálne zobrazenej časti mapy ale aj jej okolie. Okrem nich navyše na webe neposkytujú taký plnohodnotný zážitok z prezerania ako v natívnej aplikácii. Nevýhodou je aj nemožnosť posúvania webovej stránky pokiaľ je element s mapou väčší ako je rozlíšenie displeja mobilného zariadenia, lebo eventy sú zachytávané mapou a posúva sa tá.

Používateľ nemusí chcieť okamžite interagovať s mapou a v takomto prípade ho zbytočne zaťažuje. Výhodnejšie je tak zobraziť len náhľad mapy, ktorý sa načíta rýchlejšie pretože šetrí prenášané dáta, a pokiaľ sa používateľ chce dozvedieť viac, tak len jednoducho na mapu klikne. V takomto prípade sa mu automaticky načíta plne interaktívna mapa. Na zobrazenie náhľadu mapy existuje API v službe google maps\footnote{\url{https://developers.google.com/maps/documentation/staticmaps/}}, takže je možné využiť priamo to. Nevýhodou je, že počet požiadaviek za deň, ktoré sú zadarmo, je obmedzený, po získaní API kľúča je možné ich spraviť 25000, ďalšie sú spoplatňované.

Rozšírením na mobilných zariadeniach iOS je možnosť otvorenia mapy priamo v natívnej aplikácii, ktorá prináša ešte väčší používateľský zážitok ako zobrazenie na webe. To je uskutočňované pomocou url schémy. V starších verziách iOS sa nachádzala natívna aplikácia na Google API a bola vyvolaná otvorením nasledujúceho odkazu, kde bolo možné zadávať parametre zobrazenia.

\begin{verbatim}
http://maps.google.com/
\end{verbatim}

S príchodom iOS 6 bola aplikácia nahradená vlastnom Apple aplikáciou, na ktorej spustenie sa zmenila aj url schéma a pôvodná otvorí mapu len v prehliadači. Výhodou je, že na starších, respektíve nepodporovaných zariadeniach nastáva presmerovanie na stránky Google a tak na rovnakom odkaze funguje spúšťanie aj pôvodnej aplikácie. Nová url schéma má nasledujúci tvar:

\begin{verbatim}
http://maps.apple.com/
\end{verbatim}

% subsubsection mapy (end)


\subsubsection{Lightboxy} % (fold)
\label{subsub:lightboxy}

Lightbox je technika umožňujúca zobrazovať webový obsah v modálnych oknách nachádzajúcich sa nad úrovňou pôvodnej stránky.

Hlavným problémom použitia ,,lightboxov'' na mobilných zariadeniach je nesprávne zobrazovanie stránok pokiaľ nové okno je väčšie ako displej. V takomto prípade by bolo vhodnejšie používateľa presmerovať priamo na novú stránku.

% subsubsection lightboxy (end)

\subsubsection{Otvorenie v aplikácii / Stiahnutie aplikácie} % (fold)
\label{ssub:otvorenie_v_aplik_cii}

V súčasnosti sme opklopovaný množstvom informačných zdrojov, ktoré sa zväčša nachádzajú na internete dostupné na konkrétnej webovej adrese. Na tieto zdroje existuje množstvo odkazov z rôznych webových stránok, sociálnych sietí či mobilných aplikácii, ktoré zobrazia ich obsah v prehliadači.

\begin{fancybox}
\textit{,,Links don’t open apps.'' Jason Grigsby} \cite{links}
\end{fancybox}

Webové odkazy neotvárajú natívne aplikácie, čo je technicky pravda, no realita je trochu odlišná. Prepojenie webovej časti aplikácie s natívnou umožňuje otvárať komplexnejšie úlohy, na ktoré je vo webe nedostatočny výkon, priamo v natívnom kóde. To umožní plynulejší chod aplikácie a lepší používateľský zážitok.

Prepájanie aplikácii sa uskutočňuje pomocou ,,custom url schémy'' \cite{urlscheme}, ktorá je špecifická pre jednotlivé operačné systémy zariadení. Nevýhodou vlastných odkazov je, že pokiaľ používateľ nemá nainštalovanú aplikáciu, tak po kliknutí sa objaví celkom nepekná chybová hláška.

\begin{figure}[H]
	\centering
	\includegraphics[width=0.35\textwidth]{img/customerror.png}
	\caption[Chybová hláška po otvorení custom URL schémy, pokiaľ aplikácia neexistuje]{
		Chybová hláška po otvorení custom URL schémy, pokiaľ aplikácia neexistuje \cite{customscheme}.\\
		Prevzaté z http://www.lukew.com/ff/entry.asp?1654}
	\label{fig: customerror}
\end{figure}

Lepším riešením je detekovanie, či si používateľ stiahol natívnu aplikáciu a odkaz na otvorenie v nej vytvoriť až potom, čo je overenie úspešné. Pokiaľ by nebolo, tak by sa mohlo zobraziť tlačítko na stiahnutie natívnej aplikácie, ktoré opäť musí byť prispôsobené platforme na ktorej na stránku prispôsobujeme, pokiaľ nechceme používateľa zahltitť všetkými platformami ktoré podporujeme.

Samotný proces detekcie či má používateľ nainštalovanú našu aplikáciu vôbec nie je triviálny, pretože na webe neexistuje žiadne dostupné API, ktoré by zahrňovalo všetky platformy. Apple síce vydal možnosť otvorenia aplikácie pomocou ,,Smart Banners'', ale až v iOS 6 a tak táto možnosť nie je úplne použiteľná.

\begin{figure}[H]
	\centering
	\includegraphics[width=0.5\textwidth]{img/smartappbanner.png}
	\caption[Otvorenie natívnej aplikácie v iOS 6]{
		Otvorenie natívnej aplikácie v iOS 6 \cite{smartappbanner}.\\
		Prevzaté z http://developer.apple.com/}
	\label{fig: smartappbanner}
\end{figure}

Navyše ,,smart banner'' je priamo definovaný v html meta tagu aplikácie. Na stránke existuje len jeden ktorý volá url schému aplikácie a aj ten má prednastevený vzhľad v podobne okna v hornej časti obrazovky. Keby chceme vlastnú natívnu aplikáciu zavolať z viacerých častí webovej, prípadné volať viacero natívnych aplikácií, tak toto riešenie je nepostačujúce. Nemôže byť upravané vlastným potrebám.

Jediné možné riešenie je len v spolupráci s natívnou aplikáciou, aj tak sa však musí vyvolať otvorenie stránky v prehliadači, kde sa dá už do cookies alebo lokálneho úložiska programátorksy zapísať existencia aplikácie. Vytvorenie samotného ,,webView'' komponentu s webovou stránkou vrámci aplikácie a následné zatvorenie nestačí. Prehliadač má oddelené úložiská stránok pre rôzne typy prístupov ako sú s internetový prehliadač, aplikácia a v iOS aj pre otvorenie internetovaj stránky z plochy. Možnosť ako tento proces zamaskovať je skrytá v procese registrácie, keď po otvorení aplikácie a zaregistrovaní pošleme používateľovi e-mail s potvrdzujúcim odkazom, ktorý otvorí webovú stránku v prehliadači.


% subsubsection otvorenie_v_aplik_cii (end)

% subsection komponenty (end)

\subsection{Nástroje} % (fold)
\label{sub:n_stroje}

\subsubsection{Detekcia} % (fold)
\label{ssub:detekcia}

Pre potreby rozpoznávania platformy v aplikácii bol vytvorený modul detekcie podľa jedinečného ,,user agent'' textu. Modul umožňuje detekovať iOS a Android zariadenia. Rozpoznávanie prebieha na základe regulárneho výrazu, kde pri metóde detekcie iOS zariadenia sa kontroluje výsky slov ,,iPhone'', ,,iPad'' alebo ,,iPod'', ktoré jedinečne definujú platformu. Pri Android zariadeniach je detekcia jednoduchšia, pretože stačí ak sa v identifakačnom texte nachádza slovo ,,Android''.

Pri konfigurácii webovej aplikácie je možné aj aktálny ,,user agent'' text nahradiť vlasnou hodnotou, takže prehliadač sa vždy môže správať ako požadované zariadenie.

% subsubsection detekcia (end)

% subsection n_stroje (end)

% section n_stroj_na_overenie (end)

\section{Overenie} % (fold)
\label{sec:overenie}

% section overenie (end)

\section{Záver} % (fold)
\label{sec:z_ver}

% section z_ver (end)